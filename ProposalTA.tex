%! TEX program = xelatex
% 
% Template Proposal Tugas Akhir
% Program Studi Sistem dan Teknologi Informasi
% Sekolah Teknik Elektro dan Informatika
% Institut Teknologi Bandung
% 
% Dibuat oleh: IGB Baskara Nugraha 
% Email: baskara@itb.ac.id 
% 
% Last updated: 20 Oktober 2025
%
% Petunjuk penggunaan:
% 1. Ada 2 file utama, yaitu ProposalTA.tex (file ini) dan daftar-pustaka.bib (file daftar pustaka).
% 2. Sunting ProposalTA.tex sesuai dengan kebutuhan Anda.
% 3. Sunting atau generate isi daftar-pustaka.bib dengan referensi yang Anda gunakan, sesuai dengan format BibLaTeX.
% 4. Simpan kedua file tersebut dalam satu folder yang sama.
% 5. Kompilasi file ProposalTA.tex menggunakan XeLaTeX dan Biber (lihat urutan cara kompilasi di bawah).
% 6. Hasil kompilasi adalah file ProposalTA.pdf yang siap dicetak.
% 
% Urutan cara kompilasi (melalui command line):
% 1. xelatex ProposalTA.tex
% 2. biber ProposalTA      
% 3. xelatex  ProposalTA.tex
% 4. xelatex  ProposalTA.tex
%
% Catatan:
% - Pastikan Anda telah menginstal paket-paket LaTeX yang diperlukan, termasuk
%   biblatex-chicago dan fontspec.
% - Gunakan editor LaTeX yang mendukung XeLaTeX, seperti TeXstudio, Overleaf, atau lainnya.
% - Jika meenggunakan Visual Studio Code sebagai editor, pastikan mengatur "latex-workshop.latex.tools" dan
%   "latex-workshop.latex.recipes" untuk mendukung XeLaTeX dan Biber dengan cara menambahkan konfigurasi berikut:
%   "latex-workshop.latex.tools": [ 
%       {
%           "name": "xelatex",
%           "command": "xelatex",
%           "args": [
%               "-synctex=1",
%               "-interaction=nonstopmode",
%               "-file-line-error",
%               "%DOC%"
%           ]
%       },
%       {
%           "name": "biber",
%           "command": "biber",
%           "args": [
%               "%DOCFILE%"
%           ]
%       }
%   ],
%   "latex-workshop.latex.recipes": [
%       {
%           "name": "xelatex -> biber -> xelatex*2",
%           "tools": [
%               "xelatex",
%               "biber",
%               "xelatex",
%               "xelatex"
%           ]
%       }
%   ]
% - Untuk referensi lebih lanjut tentang penggunaan BibLaTeX dengan gaya Chicago, silakan merujuk ke dokumentasi resmi BibLaTeX.
%   https://ctan.org/pkg/biblatex-chicago
% - Untuk referensi lebih lanjut tentang penggunaan XeLaTeX dan fontspec, silakan merujuk ke dokumentasi resmi fontspec.
%   https://ctan.org/pkg/fontspec
% - Selamat menyusun proposal tugas akhir Anda!
%
\documentclass[12pt,a4paper,oneside]{book}

% ==========================================
% BASIC PACKAGES
% ==========================================
\usepackage[utf8]{inputenc} % for UTF-8 encoding
\usepackage{fontspec} % for font selection
\setmainfont{Times New Roman} % set main font to Times New Roman
\usepackage[a4paper, left=4cm, right=3cm, top=3cm, bottom=3cm]{geometry} % set page margins
\usepackage[indonesian]{babel} % untuk bahasa Indonesia
\usepackage{csquotes} % for context-sensitive quotation facilities
\usepackage{setspace} % for line spacing
\onehalfspacing % spasi 1.5
\usepackage{graphicx} % for images
\usepackage{caption} % for customizing captions
\usepackage{subcaption} % for sub-figures
\usepackage{hyperref} % for hyperlinks
\usepackage{titlesec} % for customizing titles
\usepackage{tocloft} % for customizing table of contents
\usepackage{lipsum} % for dummy text (lorem ipsum text)
\usepackage{floatrow} % for customizing float (figure and table) positions
\usepackage{listings} % for code listing
\usepackage{amsmath} % for math
\usepackage{amssymb} % for math symbols

\setcounter{tocdepth}{4} % kedalaman daftar isi sampai subsubbab
\setcounter{secnumdepth}{4} % kedalaman penomoran sampai subsubbab

% ==========================================
% UBAH NAMA-NAMA DEFAULT KE BAHASA INDONESIA
% ==========================================
\addto\captionsindonesian{%
  \renewcommand{\contentsname}{Daftar Isi}%
  \renewcommand{\listfigurename}{Daftar Gambar}%
  \renewcommand{\listtablename}{Daftar Tabel}%
  \renewcommand{\bibname}{Daftar Pustaka}%
  \renewcommand{\indexname}{Indeks}%
  \renewcommand{\figurename}{Gambar}%
  \renewcommand{\tablename}{Tabel}%
  \renewcommand{\partname}{Bagian}%
  \renewcommand{\chaptername}{Bab}%
  \renewcommand{\appendixname}{Lampiran}%
  \renewcommand{\abstractname}{Abstrak}%
}

% ==========================================
% SITASI DAN DAFTAR PUSTAKA (MENGGUNAKAN CHICAGO MANUAL OF STYLE)
% ==========================================
% --- Bibliografi: biblatex + biber ---
\usepackage[
  backend=biber,
  style=authoryear,
  citestyle=authoryear,
  date=year,
  maxcitenames=2,
  uniquename=false
]{biblatex}
\addbibresource{daftar-pustaka.bib}

% Hilangkan kutip di judul
\DeclareFieldFormat{title}{#1}
\DeclareFieldFormat[article]{title}{#1}
\DeclareFieldFormat[incollection]{title}{#1}
\DeclareFieldFormat[inbook]{title}{#1}
\DeclareFieldFormat[book]{title}{#1}
\DeclareFieldFormat[online]{title}{#1}
\DeclareFieldFormat[techreport]{title}{#1}

% Pastikan \cite() menampilkan (Penulis Tahun)
\let\oldcite\cite
\renewcommand{\cite}{\parencite}

\DeclareLanguageMapping{indonesian}{english}

% ==========================================
% Ubah istilah bahasa Inggris di daftar pustaka ke Bahasa Indonesia
% ==========================================
\DefineBibliographyStrings{english}{
  and          = {dan},
  andothers    = {dkk.},
  editor       = {penyunting},
  editors      = {penyunting},
  translator   = {penerjemah},
  byeditor     = {disunting oleh},
  bytranslator = {diterjemahkan oleh},
  in           = {dalam},
  edition      = {edisi},
  pages        = {hal.},
  page         = {hal.},
  volume       = {vol.},
  number       = {no.},
  urldate      = {diakses pada},
  url          = {tautan},
}

\renewbibmacro*{in:}{%
  \printtext{Dalam\addcolon\space}}

\renewbibmacro*{byeditor+others}{%
  \ifnameundef{editor}
    {}
    {\usebibmacro{byeditor+othersstrg}%
     \setunit{\addspace}%
     \printnames[byeditor]{editor}%
     \clearname{editor}%
     \newunit}%
  \usebibmacro{byeditorx}%
  \usebibmacro{bytranslator+others}}

\renewbibmacro*{byeditor+othersstrg}{%
  \printtext{Disunting oleh}}

% Override format urldate, pages, URL, edition
\DeclareFieldFormat{urldate}{\mkbibparens{diakses pada\space\mkbibdatelong{urldate}}}
\DeclareFieldFormat{pages}{hal\adddot\space#1}
\DeclareFieldFormat{url}{URL\addcolon\space\url{#1}}
\DeclareFieldFormat{edition}{%
  \ifinteger{#1}
    {Edisi~#1}
    {#1}}

% Override format urldate, pages, dan URL
\DeclareFieldFormat{urldate}{\mkbibparens{diakses pada\space#1}}
\DeclareFieldFormat{pages}{hal\adddot\space#1}
\DeclareFieldFormat{url}{URL\addcolon\space\url{#1}}  % ← URL huruf besar
\DeclareFieldFormat{postnote}{hal\adddot\space#1}


% Atur pemisah nama penulis
\renewcommand*{\multinamedelim}{\addcomma\space}
\renewcommand*{\finalnamedelim}{\addspace dan\space}
\renewcommand*{\finalandcomma}{}

\ExecuteBibliographyOptions{
  maxbibnames=6,   % berapa nama di daftar pustaka sebelum 'dkk.'
  dashed=false     % hilangkan tanda '—' untuk author yang sama berurutan
}

% ==========================================
% Atur pemisah nama penulis agar lebih natural dalam Bahasa Indonesia
% ==========================================
\renewcommand*{\finalandcomma}{} % hilangkan koma sebelum 'dan'


% ==========================================
% TAMPILAN
% ==========================================
\hypersetup{
    colorlinks=true,
    linkcolor=black,
    citecolor=black,
    urlcolor=black
}

% -- No Header dan No Footer ---
\pagestyle{plain}

% --- Ubah nama daftar listing ke "DAFTAR ALGORITMA" ---
% --- Harus diletakkan sebelum \begin{document} ---
\renewcommand{\lstlistlistingname}{\centering\normalsize DAFTAR ALGORITMA} 
\renewcommand{\lstlistingname}{Algoritma}
\captionsetup[lstlisting]{justification=raggedright,singlelinecheck=false}


\renewcommand \cftchapdotsep{4.5}

% ==========================================
% AWAL DOKUMEN
% ==========================================
\usepackage{indentfirst}
\begin{document}

% ==========================================
% HALAMAN JUDUL
% ==========================================
\begin{titlepage}
\begin{center}
    
    {\large\bfseries PERANCANGAN SISTEM \textit{BUSINESS INTELLIGENCE} BERBASIS \textit{CHATBOT} DENGAN \textit{RULE-BASED QUERY}, \textit{NATURAL LANGUAGE PROCESSING}, DAN \textit{TIME SERIES FORECASTING} UNTUK ANALISIS DAN PREDIKSI PERMINTAAN LAYANAN PELANGGAN}\\
    
    
    {\large Oleh}\\[0.3cm]
    \textbf{
    {\large Thalita Zahra Sutejo}\\
    {\large 18222023}
    }\\

    \vspace{2cm}
    
    \begin{figure}[h]
    \centering
    \includegraphics[width=0.2\textwidth]{ganesha.jpg}
    \end{figure}
    
     \vspace{4cm}

    \textbf{
    {\large PROGRAM STUDI SISTEM DAN TEKNOLOGI INFORMASI}\\
    {\large SEKOLAH TEKNIK ELEKTRO DAN INFORMATIKA}\\
    {\large INSTITUT TEKNOLOGI BANDUNG}\\
    {\large 2025}
    }
\end{center}
\end{titlepage}



% ==========================================
% LEMBAR PENGESAHAN 
% ==========================================
\newpage
\thispagestyle{empty}
\pagenumbering{gobble}
\begin{center}
  \textbf{\large LEMBAR PENGESAHAN}\\[1cm]
  \vspace*{1.5cm}
    
  {\large\bfseries PERANCANGAN SISTEM \textit{BUSINESS INTELLIGENCE} BERBASIS \textit{CHATBOT} DENGAN \textit{RULE-BASED QUERY}, \textit{NATURAL LANGUAGE PROCESSING}, DAN \textit{TIME SERIES FORECASTING} UNTUK ANALISIS DAN PREDIKSI PERMINTAAN LAYANAN PELANGGAN}\\
     \vspace{2cm}

  {\Large \textbf{Proposal Tugas Akhir}}\\


  \vspace{1.5cm}
    
    
  {\large Oleh}\\[0.3cm]
    \textbf{
    {\large Thalita Zahra Sutejo}\\
    {\large 18222023}
  }\\
    
  \vspace{0.5cm}
 
  {\large Program Studi Sistem dan Teknologi Informasi}\\
  {\large Sekolah Teknik Elektro dan Informatika}\\
  {\large Institut Teknologi Bandung}\\

  \vspace{1.5cm}

  Proposal Tugas Akhir ini telah disetujui dan disahkan\\ 
  di Bandung, pada tanggal 22 Oktober 2025\\[1cm]

% ==========================================
% Versi 1 pembimbing (default)
% ==========================================
	Pembimbing  \\[3cm]
	Dr. Ir. Dimitri Mahayana, M.Eng.   \\[0.2cm]
	NIP. 196808091991021001 
% ==========================================

\end{center}

\vspace{1cm}
\noindent

% ==========================================
% Jika ada 2 pembimbing TA, uncomment dan edit 
% tabular di bawah ini. Kemudian, comment out atau hapus
% bagian versi 1 pembimbing di atas.
% ==========================================

%\begin{tabular}{p{1cm}p{7cm}p{7cm}}
%   & Pembimbing 1 & Pembimbing 2 \\[3cm]
%   & Dr. Ir. John Doe, M.T. & Dr. Mary Doe, M.Sc. \\[0.2cm]
%   &  NIP. 123456789 & NIP. 987654321
%\end{tabular}


% -- Change page number style to roman ---
\pagenumbering{roman} 


% ==========================================
% DAFTAR ISI, TABEL, GAMBAR
% ==========================================
% --- DAFTAR ISI ---
\makeatletter
\renewcommand{\tableofcontents}{%
  \clearpage
  \thispagestyle{plain}% no header
  \begin{center}
    {\large\bfseries\MakeUppercase{Daftar Isi}\par}
  \end{center}
  \vskip 1em
  \@starttoc{toc}%
}
\makeatother

\newpage
\renewcommand{\cfttoctitlefont}{\hfill\large\bfseries\MakeUppercase}
\renewcommand{\cftaftertoctitle}{\hfill}
\tableofcontents
%\addcontentsline{toc}{chapter}{DAFTAR ISI}

\mainmatter
% --- FORMAT TAMPILAN JUDUL BAB, SUBBAB, JUDUL GAMBAR DAN TABEL ---

% --- Judul Bab ---
\renewcommand\thechapter{\Roman{chapter}}
\titleformat{\chapter}[display]
  {\centering\normalfont\large\bfseries} % Format keseluruhan judul bab
  {\MakeUppercase{\chaptertitlename}\ \thechapter}{0pt}{\large} % Format nomor bab
\titlespacing*{\chapter}{0pt}{-1.5em}{1.5em} % <--- atur jarak atas dan bawah judul bab di sini

% --- Judul Subbab dan Subsubbab ---
\titleformat{\section}
  {\normalfont\bfseries}
  {\thesection}{1em}{}
\titlespacing*{\section}{0pt}{1.2em}{0.5em}

\titleformat{\subsection}
  {\normalfont\bfseries}
  {\thesubsection}{1em}{}
\titlespacing*{\subsection}{0pt}{0.8em}{0.3em}

% --- Format judul gambar dan tabel ---
\captionsetup[figure]{labelsep=space}
\captionsetup[table]{labelsep=space}
\floatsetup[table]{capposition=top}
\captionsetup[lstlisting]{labelsep=space}

% --- Atur indentasi paragraf ---
\setlength{\parindent}{0pt}

% --- Ganti gaya penomoran halaman menjadi arab ---
\pagenumbering{arabic} 

% ==========================================
% BAB I PENDAHULUAN
% ==========================================
\chapter{PENDAHULUAN}

% --- Latar Belakang ---
\section{Latar Belakang}

% Pengaturan spasi & tab paragraf (indent + jarak antaraparagraf)
\setlength{\parindent}{1.25cm}
\setlength{\parskip}{6pt}

Transformasi digital mendorong organisasi mengelola data dengan tiga ciri utama, yaitu \textit{volume} besar, \textit{velocity} tinggi, dan \textit{variety} format (3V). Kerangka 3V menjelaskan mengapa pendekatan analitik konvensional sering kewalahan ketika berhadapan dengan data modern yang terus bertambah, bergerak cepat, dan heterogen \parencite{Laney2001}. Di tingkat konseptual, diskursus \textit{big data} menekankan bahwa perubahan bukan hanya pada skala, melainkan juga pada cara data diproduksi, diatur, dan dimaknai lintas konteks sosial ekonomi \parencite{Kitchin2016}.

Dalam konteks Indonesia, PT Telkom Indonesia melalui \textit{IndiBiz} memosisikan diri sebagai ekosistem solusi digital bagi segmen B2B, pelaku UMKM, institusi pendidikan, dan badan usaha lokal. Lini produk ini memadukan layanan konektivitas dan solusi \textit{software} untuk mendukung operasi bisnis \parencite{IndiBiz2024Site}, serta memperoleh sorotan positif pada agenda digitalisasi UMKM \parencite{Telkom2024News}. Kebutuhan analitik yang cepat dan inklusif menjadi krusial agar pengambilan keputusan selaras dengan dinamika pasar dan kebutuhan pelanggan.

Dalam praktik, proses analitik di banyak organisasi besar masih manual dan terfragmentasi. Pengguna perlu membuka beberapa \textit{dashboard}, menyelaraskan definisi \textit{key performance indicator} (KPI), menormalkan wilayah dan periode data, lalu menulis kueri \textsc{sql} ke gudang data. Literatur \textit{business intelligence} tingkat manajerial menempatkan persoalan ini sebagai hambatan terhadap \textit{time to insight} dan konsistensi makna indikator \parencite{Sharda2020}. Upaya \textit{self-service} BI (SSBI) yang idealnya memandirikan pengguna nonteknis sering terkendala akses dan kualitas data, ketergantungan pada peran teknis, serta faktor organisasi yang perlu dikelola sistematis \parencite{Lennerholt2021,Lennerholt2023}. Tumpukan \textit{dashboard} dan laporan pada akhirnya dapat memperpanjang jarak antara pertanyaan bisnis dan jawaban yang dapat ditindaklanjuti.

Salah satu pendekatan untuk mengurangi hambatan adalah antarmuka bahasa alami yang memungkinkan pengguna mengajukan pertanyaan langsung ke data tanpa menulis \textsc{sql}. Bidang \textit{natural language to} SQL (\textit{text-to-}SQL atau NL2SQL) menunjukkan kemajuan signifikan. Kajian mutakhir memperlihatkan peran \textit{large language model} (LLM) dalam meningkatkan cakupan dan ketepatan pemetaan semantik, sekaligus menyoroti kebutuhan jaminan kualitas eksekusi, validitas keluaran, ketahanan terhadap variasi bahasa, dan strategi pascaproses yang kokoh pada lingkungan \textit{enterprise} \parencite{Gao2023,Li2024}. Tantangan NL2SQL karenanya bukan semata pemodelan, melainkan rekayasa sistem ujung ke ujung.

Keterbatasan lain yang kerap dijumpai ialah dominasi analitik deskriptif yang hanya menjawab apa yang terjadi. Analitik prediktif, khususnya \textit{time series forecasting}, terbukti bernilai untuk perencanaan kapasitas, estimasi permintaan, pengaturan persediaan, dan mitigasi \textit{churn}. Survei komprehensif menempatkan arsitektur \textit{deep learning} sebagai pilar yang kian lazim untuk meningkatkan akurasi peramalan lintas domain \parencite{Lim2021,Benidis2020}. Integrasi kapabilitas prediktif ke alur BI memungkinkan organisasi bergeser dari pola reaktif menuju pola proaktif.

Cara penyajian hasil analisis juga menentukan adopsi di kalangan pemangku kepentingan nonteknis. Laporan berbentuk tabel atau angka mentah sering sulit ditafsirkan. Modul \textit{natural language generation} (NLG) dapat menerjemahkan keluaran kueri dan model menjadi narasi yang ringkas, jelas, dan kontekstual. \parencite{Gatt2018} memetakan tugas inti, aplikasi, dan evaluasi NLG secara mendalam, sedangkan tinjauan komersial menunjukkan kematangan ekosistem NLG dalam praktik industri \parencite{Dale2020}. Dengan NLG, hasil analisis dapat hadir sebagai uraian yang memandu pembaca menuju \textit{insight} dan tindakan.

Bertolak dari fondasi tersebut, celah yang hendak dijawab adalah ketiadaan solusi terpadu yang menyatukan empat kemampuan kunci dalam satu antarmuka. Kemampuan tersebut mencakup interaksi \textit{natural language} untuk bertanya, penerjemahan otomatis ke kueri yang valid (NL2SQL), analitik prediktif berbasis \textit{time series forecasting}, dan NLG untuk menghasilkan narasi interpretatif yang mudah dipahami. Dari sisi arsitektur, solusi juga perlu menghormati prinsip BI manajerial seperti konsistensi definisi metrik, tata kelola data, dan pelacakan \textit{lineage} agar layak dioperasionalkan pada skala \textit{enterprise} \parencite{Sharda2020} dan relevan dengan kebutuhan unit bisnis seperti \textit{IndiBiz} \parencite{IndiBiz2024Site}.

Rumusan masalah pada pekerjaan ini mencakup perancangan antarmuka \textit{natural language} yang inklusif bagi pengguna lintas fungsi dengan tetap menjaga validitas kueri basis data \parencite{Gao2023,Li2024}, integrasi \textit{time series forecasting} ke alur analitik untuk mengantisipasi perubahan permintaan dan kinerja \parencite{Lim2021,Benidis2020}, penyajian hasil dalam bentuk narasi yang ringkas dan kontekstual melalui NLG \parencite{Gatt2018,Dale2020}, serta orkestrasi keempat kapabilitas dalam kerangka BI yang sejalan dengan tata kelola data dan kebutuhan operasional.

% ==========================================
% DAFTAR PUSTAKA
% ==========================================
\printbibliography[heading=bibintoc, title={DAFTAR PUSTAKA}]


\end{document}